%!TEX root = ../report.tex
\subsection{The Problem} % (fold)
\label{sub:the_problem}
Sentiment analysis is the area of natural language processing (NLP) concerned with computationally identifying the emotion expressed by an author of a piece of text (positive, negative, neutral). Determining the sentiment of a piece of text has important implications for opinion mining, parsing users reviews and recommendation systems. Humans communicate in complicated ways and often don't strictly use literal language. Figurative language, such as sarcasm, irony, and metaphors, are quite prevalent in standard human communication. Hence, in order to create better representations of human language, systems must take figurative language into account.
% subsection the_problem (end)

\subsection{Related Work} % (fold)
\label{sub:related_work}

% subsection related_work (end)

\subsection{Motivation} % (fold)
\label{sub:motivation}

% subsection motivation (end)